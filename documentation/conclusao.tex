\chapter{Conclusão}
\label{conclusao}
O sistema de criptografia copletamente homomórfico descrito por Gentry resolve o problema de delegar processamento com privacidade, sem que a segurança seja comprometida. Apesar do método de construção ser um tanto extenso, o algoritmo final  tem a implementação relativamente simples, apesar de que tende a ficar mais complexo a medida que otimizações são adicionadas.

Esse novo paradigma de criptografia faz com que várias aplicações sejam possíveis e possibilita que aplicações já existentes sejam extendidas para que haja confidencialidade e privacidade (Como a computação em nuvem).

A solução atual ainda tem muito a ser melhorada para que possa ser usada em larga escala, devido as limitações de seu tempo de execução e tamanho necessário das chaves para garantir um modelo seguro. Mas podemos há avanços sendo feitos e baseado neste modelo vários outros sistema de criptografias parecidos já foram criados e publicados, cada vez mais próximo de um modelo que possa realmente ser usado.