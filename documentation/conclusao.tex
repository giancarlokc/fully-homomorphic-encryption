\chapter{Conclusão}
\label{conclusao}
O sistema de criptografia completamente homomórfico descrito por Gentry resolve o problema de delegar processamento com privacidade, sem que a segurança seja comprometida. Mesmo que o método de construção seja um tanto extenso, o algoritmo final  tem a implementação relativamente simples, apesar de que tende a ficar mais complexo a medida que otimizações são adicionadas.

Esse novo paradigma de criptografia faz com que várias aplicações sejam possíveis e possibilita que aplicações já existentes sejam extendidas para que haja confidencialidade e privacidade (como a computação em nuvem).

A solução atual ainda tem muito a ser melhorada para que possa ser usada em larga escala, devido as limitações de seu tempo de execução e tamanho necessário das chaves para garantir um modelo seguro. Apesar disso, há avanços sendo feitos, vários outros sistemas de criptografias semelhantes já foram criados baseados no modelo de Gentry. Esses novos sistemas trazem novas otimizações e estão cada vez mais próximo de um modelo viável de criptografia completamente homomórfica.