The fully homomorphic encryption (\textbf{FHE}) allows mathematical functions to be applied to encrypted data. This concept appeared after the creation of the RSA but only after several years Craig Gentry was the first to come up with a solution to this problem.
The construction of Gentry's FHE scheme starts by creating a somewhat homomorphic encryption scheme. To reduce the noise created during the homomorphic operation the decryption function is executed homomorphically in the doubly encrypted data. Finally, the decryption function is simplified to reduce noise and allow the scheme to execute the decryption function.
The security of the FHE scheme is based on parts of the algorithm that can be reduced to hard problems. While this problems remain with no efficient solution, the scheme is considered secure. One example is the approximate maximum common divisor problem.
There are several application that are possible with a fully homomorphic encryption scheme and some of them are described in this monograph: cloud computing with privacy, secure searches, software security, proxy re-encryption and secure email filters.
After tests comparing the FHE with the RSA, it's possible to see why the fully homomorphic encryption is still not ready to be fully used. But there are a lot of advances on optimizing the algorithm.

\noindent \textbf{Keywords}: Fully Homomorphic Encryption, Security, Privacy