\chapter{Segurança}
De acordo com Gentry, para que uma criptografia seja considerada segura ela deve atender dois critérios: \textit{one-wayness} e \textit{segurança semântica}~\cite{probability}.

\section{Padrões de segurança}
\textit{One-wayness} significa que, dado uma chave pública $pk$ e uma mensagem criptografada $c$ por $pk$, deve ser difícil para um adversário gerar o texto plano correspondente. Um algoritmo que rode em tempo polinomial $\lambda$ deve ter uma probabilidade de sucesso menor que $1/\lambda^k$ para uma qualquer constante $k$.

Para que um sistema seja \textit{semanticamente seguro}, dado um texto criptografado $c$ que criptografa o texto plano $m_0$ ou $m_1$, o adversário deve ter 50\% de chance de acerto. Se ele tiver mais, significa que há algum processo além de pura probabilidade. Por isso um sistema determinístico\footnote{Onde há apenas um texto criptografado para cada texto plano.} não é considerado semanticamente seguro, pois o adversário pode gerar as criptografias dos textos $m_0$ e $m_1$ e comparar com $c$. Isso nos diz que um sistema seguro deve ser probabilístico.

\section{Prova de segurança semântica}
Apesar do sistema de criptografia descrito ser "maleável" devido à função $Evaluate$ que muda a criptografia de uma chave para outra, o sistema continua sendo \textit{semanticamente seguro}. Gentry mostra isso comparando a função de criptografia com o problema de encontrar o \textit{maior divisor comum} (MDC). Euclides mostra que achar o MDC é fácil, porém na função de criptografia o que temos é MDC aproximado.
\begin{center}
	$x_1 = s_1 + p \cdot q_1$
\end{center}
Com $s_i$ muito menor que p e com um parâmetro de segurança $\lambda$, esse problema não pode ser resolvido polinomialmente com relação ao comprimento de bits de $x_1$.

Além disso, Gentry mostra que esse problema de MDC aproximado pode ser reduzido ao problema de segurança semântica do sistema. Desse modo a segurança semântica só pode ser quebrada se o problema de MDC aproximado for possível em tempo polinomial.

\section{Segurança com relação à dica dentro chaves}
Como visto previamente, para que o sistema seja completamente homomórfico uma dica da chave secreta é colocada na chave pública. Essa maleabilidade pode dar a impressão que o sistema deixaria de ser seguro, no entanto Gentry prova que o sistema continua seguro desde que:
\begin{itemize}
	\item O sistema parcialmente homomórfico que da origem ao sistema completamente homomórfico seja semanticamente seguro.
	\item Dado um conjunto de números $y$ e outro número $s$, seja difícil achar um subconjunto esparço de $y$ com somatório igual a $s$.
\end{itemize}
O último representa o problema de descobrir $p$ a partir da dica. Esse problema é chamado de Problema da Soma de Subconjunto Esparço (PSSE) e já foi estudado com relação a sistema de criptografias com servidores. Se o tamanho do vetor e seu peso Hamming forem escolhidos corretamente, ele não pode ser resolvido em tempo polinomial.