\chapter{Introdução}
\label{introducao}
A criptografia está presente em quase todas as aplicações de hoje e com ela é possível garantir que informação seja trocada de modo seguro. Há dois métodos de criptografia que são utilizados atualmente: simétrico e assimétrico.

A criptografia simétrica é o método onde o remetente e o destinatário usam a mesma chave, exemplos dessa criptografia são o DES e o AES. Porém a criptografia que realmente garante os padrões de segurança atuais é a assimétrica, onde há uma chave pública e uma chave privada. É comum que comunicações que usam criptografia simétrica façam a troca das chaves utilizando algum algoritmo assimétrico. Dos algoritmos assimétricos, o RSA é o mais utilizado e é o algoritmo que deu origem a criptografia homomórfica.

Esta monografia tem como objetivo apresentar o problema da Criptografia Completamente Homomórfica, soluções atuais, possíveis aplicações e limitações.
Essa criptografia, apesar de recente, tem a capacidade de inovar o campo da computação em nuvem, pois ela não necessita que os dados sejam descriptografados para que operações sejam aplicadas nos dados originais. Ou seja, é possível ter total privacidade.

A solução apresentada por Gentry para o problema de Criptografia Completamente Homomórfica~\cite{gentrythesis}, atualmente considerada a melhor solução para tal problema, é descrita nesta monografia com relação a criptografia sobre inteiros~\cite{fheintegers} e seu método de construção é explicado seguindo os seus passos de contrução originais e padrões de implementação utilizados em~\cite{gentryhalevi}.

Possíveis aplicações para esse sistema são discutidas, mostrando qual o papel da criptografia completamente homomórfica nelas e qual a viabilidade de implementação delas.
Uma seção é dedicada a descrever a segurança deste sistema de criptografia, verificando o que garante que nenhuma informação seja liberada quando operações são aplicadas ao texto criptografado.

Além disso, testes foram realizados utilizando uma implementação deste sistema de criptografia\footnote{A implementação escolhida é baseada na criptografia com relação a inteiros.} e seus resultados foram comparados com outro sistema de criptografia assimétrico, o RSA.

Concluímos com algumas observações sobre as diferenças entre essa criptografia e outras criptografias assimétricas, limitações atuais, possibilidades futuras e melhorias necessárias para fazer com que a criptografia completamente homomórfica possa ser usada em larga escala.