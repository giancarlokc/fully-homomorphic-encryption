A criptografia completamente homomórfica (\textbf{CCH}) permite que qualquer função matemática possa ser aplicada aos dados criptografados. Esse conceito surgiu logo após a criação do RSA porém somente anos depois Craig Gentry foi o primeiro a ter uma solução para este problema.
A construção do sistema CCH de Gentry começa com a criação de um sistema parcialmente homomórfico. Para reduzir o resíduo inserido a cada operação homomórfica a função de descriptografia é executada como operação homomórfica nos dados duplamente criptografados. Finalmente, a função de descriptografia é simplificada para que seu circuito possa ser simples o suficiente para ser executado pelo sistema.
A segurança do CCH se baseia em partes do algoritmo que podem ser reduzidas a problemas difíceis. Enquanto esses problemas continuarem a não ter um solução eficiente, o sistema é considerado seguro. Um exemplo é o problema de encontrar o máximo dividor comum aproximado.
Há várias aplicações possíveis para um sistema completamente homomórfico e alguns desses são descritos nesta monografia: computação em nuvem com privacidade, buscas seguras, segurança de software, re-criptografia de proxys e filtros de emails sigilosos.
Após testes serem feitos comparando a CCH com o RSA é possível ver porque a criptografia completamente homomórfica ainda não é usada em larga escala. No entanto, há vários avanços sendo feitos em otimizações do algoritmo.

\noindent \textbf{Palavras-chave}: Criptografia Completamente Homomórfica, Segurança, Privacidade
