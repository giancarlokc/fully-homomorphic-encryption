\chapter{Testes e Resultados}
\label{testes}
\section{Implementação do algoritmo}
\label{resultados}
A implementação usada para testes com criptografia completamente homomórfica foi a biblioteca FHEW\footnote{Pode ser encontrada em https://github.com/lducas/FHEW} criada por Leo Ducas e Daniele Micciancio baseada em~\cite{fhew}.

A biblioteca ofereçe funções para:
\begin{itemize}
	\item Criar chave secreta e chave pública
	\item Criptografar bit
	\item Descriptografar bit criptografado
	\item Executar um NAND homomórfico em dados criptografados
\end{itemize}

\section{Arquitetura usada}
A arquitetura do computador usado para os testes é a seguinte:
\begin{itemize}
	\item \textbf{Processador}: Intel(R) Core(TM) i5-3337U CPU @ 1.80GHz
	\item \textbf{Memória RAM}: 8,00 GB (7,88 GB utilizáveis)
	\item \textbf{Sistema Operacional}: Debian 7 (64 bits)
\end{itemize}

\section{Algoritmo usado para comparação}
O algoritmo RSA foi usado como comparação pois ele é usado em larga escala atualmente, além de ser a origem da CCH. A implementação utilizada foi a biblioteca Crypto++\footnote{Pode ser encontrada em www.http://cryptopp.com/} por ser próxima a um RSA puro (sem muitas otimizações) e por ser na mesma linguagem de programação que a implementação do CCH. Desse modo a compilação pode ser feita com o mesmo compilador e nível automático de otimização (-O1).
	
\section{Metodología}
É possivel dividir a comparação em três etapas: velocidade, segurança e funcionalidades.

Para comparar a velocidade dos dois tipos de criptografia os testes foram realizados em um conjunto de 100 strings de 80 bits cada (10 caracteres). Cada string é criptografada usando RSA e CCH, e em seguida ela é descriptografada. O tempo médio das operações do RSA para as 100 strings é calculado e comparado com as do CCH.
	
Para comparar a segurança podemos olhar para estratégias comuns de ataques e verificar a probabilidade de tal ataque ser bem sucedido para cada criptografia.
	
Para comparar a funcionalidade olhamos para o que a criptografia pode oferecer, e baseado nas comparações anteriores quais são os pontos fortes e pontos fracos de cada uma.
	
\section{Velocidade}
É possivel ver que o sistema atual de criptografia completamente homomórfica é muito mais lento do que o RSA. Os tempos obtidos nos testes são podem ser vistos na tabela \ref{tab:LABEL_TAB_RESULTADOS}.
	
	\begin{table}[!h]
  	\centering
  	\begin{tabular}{ |l|l|l|l| }
    	\hline
      	Sistema & Criptografia & Descriptografia & Operação NAND \\
    	\hline
      	RSA & 21 Kb/s & 21 Kb/s & Não disponível \\
    	\hline
      	CCH & 214 b/s & 188 b/s & 0,46 b/s \\
    	\hline
  	\end{tabular}
  	\caption{Média dos tempos dos testes}
  	\label{tab:LABEL_TAB_RESULTADOS}
	\end{table}

	Os tempos para criptografar e descriptografar são muito mais lentos no CCH, porém o que faz com que a CCH não possa ser usada em larga escala é o tempo para fazer operações homomórficas. É possivel ver avanços que estão sendo feitos para tornar essa criptografia mais utilizável. Várias possíveis otimizações são descritas em~\cite{optone} e~\cite{opttwo} que mostram como diminuir o tempo para criptografar e da função $Recrypt$ (Para deixar as operações mais rápidas). Também é possível notar que o tamanho da chave pública nesta implementação tem aproximadamente 12GB, mas uma implementação do algoritmo encontrada no Github sugere a possibilidade de usar uma chave comprimida com 2MB~\cite{coron}.
	
\section{Segurança}
	Tanto a segurança do RSA quanto da CCH é baseada em problemas matemáticos e no fato de eles serem difíceis (sem solução eficiente).
	A CCH tem sua segurança baseada no problema de MDC aproximado e no PSSE, que já foram mencionados previamente. O RSA também tem sua segurança baseada em dois problemas, o de fatoração de números grandes\footnote{Fatoração de números grandes é a decomposição de um inteiro composto grande em um produto de números menores} e o problema RSA\footnote{O problema RSA pode ser definido como encontrar a $e$-raiz de um número aleatório}.
	
	Comparar a complexidade destes problemas não está no escopo desta monografia, mas é possível dizer que os dois sistemas de criptografia são seguros.
	
\section{Funcionalidade}
	O RSA é usado na maioria das aplicações que usam criptografia assimétrica, além de possibilitar assinaturas digitais e é considerado um dos algorimos assimétricos mais seguros já construídos.
	Mas na questão de funcionalidade, apesar do RSA ser muito presente hoje em dia, a CCH é muito mais versátil e torna muitas aplicações possíveis pois ele mantém a segurança do RSA mas o extende para ter total sigilo\footnote{Por total sigilo se quer dizer que nem quem faz operações nos dados consegue vê-los} sobre os dados criptografados.
	Obviamente cada criptigrafia tem um uso diferente mas é possível ver que a CCH tem a capacidade de revolucionar o campo da criptografia, assim como o RSA fez quando foi criado.